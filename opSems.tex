 \newcommand{\shuffle}{\,{{\sqcup}\hspace{-1.5pt}{\sqcup}}\,}
 \newcommand{\slang}{\mathtt{L}}
 \newcommand{\filter}[2]{#1_{\mid #2}}
\mkfun{\ssplit}{split}{}
\mkfun{\join}{join}{}
\mkfun{\subject}{subj}{}
\mkfun{\ssynch}{ss}{}
\mkfun{\dsynch}{ds}{}
\mkfun{\psynch}{ps}{}
\newcommand{\gasem}[1]{\asem{#1}_{\mathbf{U}}}
\newcommand{\commds}{\comm_{\!\mathsf{ds}}}




We briefly recall a few notions used through the paper.
%
The sets of finite and infinite words on a given alphabet $\Sigma$ are,
respectively, denoted by $\Sigma^\star$ and $\Sigma^\omega$ (recall
that an infinite word, aka $\omega$-word~\cite{Staiger97}, on $\Sigma$
is a map from natural numbers to $\Sigma$).
%
Let $\_\cat\_$ be the concatenation operator on words and $\epsilon$
its neutral element.
%
We write $a_0 \cat a_1 \cat a_2 \cat \ldots$ for the word mapping $i$
to $a_i \in \Sigma$ for all natural numbers $i$.
%
A language $L$ on $\Sigma$ is a subset of $\Sigma^\infty$ where
$\Sigma^\infty = \Sigma^\star \cup \Sigma^\omega$.  The prefix-closure
of $L \subseteq \Sigma^\infty$ is
$\pref[L] = \Set{\aword \in \Sigma^\infty \sst \exists \aword' \in L
  \qst \aword \preceq \aword'}$, where $\preceq$ is the prefix
relation.
%
As usual we shall write $\aword\prec\aword'$ whenever
$\aword\preceq\aword'$ and $\aword\neq\aword'$.
%
Recall that $L \subseteq \Sigma^\infty$ is \emph{prefix-closed} if
$L = \pref[L]$.
%
A word $\aword$ is \emph{maximal} in a language
$L \subseteq \Sigma^\infty$ if whenever $\aword \preceq \aword'$ for
$\aword' \in L$ then $\aword' = \aword$.

We shall deal with languages on particular alphabets, namely the
alphabets of \emph{interactions} $\alfint$ and of \emph{actions}
$\alfact$, defined as follows\footnote{These sets may be infinite;
  formal languages over infinite alphabets have been studied, e.g.,
  in~\cite{ajb80}.}
%
%\footnote{Following customary notation, given
%  an alphabet $\Sigma$, $\Sigma^\star$ and $\Sigma^\omega$
%  respectively denote the set of finite and infinite words on
%  $\Sigma$; let $\Sigma^\infty = \Sigma^\star \cup \Sigma^\omega$ be
%  the $\omega$-words on $\Sigma$.
%  %
%  Also, $\prec$ is the prefix relation on words and
%  $\pref[L] = \Set{w \in \Sigma^\infty \sst \exists w' \in L \qst w
%	 \prec w'}$ is the prefix-closure of language $L \subseteq \Sigma^\infty$.
%  %
%  Recall that $L \subseteq \Sigma^\infty$ is \emph{prefix-closed} if
%  $L = \pref[L]$.
%  %
%} 
\begin{align*}
  \alfint & = \Set{\gint[] \sst \p \neq \q \in \mathfrak{P}, \msg \in \mathfrak{M}}
  & & \text{ranged over by $\aint, \aint[b], \ldots$}
  \\
  \alfact & = \Set{\aout, \ain \sst \ptp[A] \neq \ptp[B] \in \mathfrak{P}, \msg \in \mathfrak{M}}
  & & \text{ranged over by $\aact, \aact[b], \ldots$}
\end{align*}
%
where $\mathfrak{P}$ is a fixed set of \emph{participants} (or
\emph{roles}, ranged over by $\p$, $\ptp[B]$, etc.) and $\mathfrak{M}$
is a fixed set of \emph{messages} (ranged over by $\msg$, $\msg[x]$,
etc.); we take $\mathfrak{P}$ and $\mathfrak{M}$ disjoint.
%
Let $\msgof[\gint] = \msgof[\aout] = \msgof[\ain] = \msg$ and
% , $\sndint[\aint]$ its \emph{sender} \p, and $\rcvint[\aint]$ its \emph{receiver} \q; we set
$\ptpof[\gint] = \ptpof[\aout] = \ptpof[\ain] = \Set{\p,\q}$.
%
These functions extend homomorphically to (sets of) words.
%
The \emph{subject} of $\aout$ is the sender \p\ and the subject of
$\ain$ is the receiver \q.
%
Words on $\alfint^\infty$ (ranged over by $\acword,\acword',...$) are
called \emph{interaction words} while those on $\alfact^\infty$
(ranged over by $\aaword,\aaword',...$) are called \emph{words of
  actions}.
%
We let $\aword,\aword',...$ range over
$\alfint^\infty \cup \alfact^\infty$ and use $\clang$ and $\alang$ to
range over subsets of, respectively, $\alfint^\infty$ and
$\alfact^\infty$.

A \emph{global language} specifies the expected interactions of a
system while a \emph{local language} specifies the communication
behaviour of participants.

\begin{definition}[Global language]\label{def:chorlang}
  A \emph{global language} (\emph{g-language} for short) is a
  prefix-closed language $\clang$ on $\alfint^\infty$ such that
  $\ptpof[\clang]$ is finite.
\end{definition}

\begin{definition}[Local language]\label{def:actlang}
  A \emph{local language} (\emph{l-language} for short) is a
  prefix-closed language $\alang$ on $\alfact$ such that 
  $\ptpof[\alang]$ is finite.
  %
  An l-language is \emph{$\p$-local} if its words have all actions
  with subject \p.
\end{definition}

%
As discussed in \cref{sec:intro}, l-languages give rise to
\emph{communicating systems}.
\begin{definition}[Communicating system]\label{def:commSyst}
  Let $\ptpset \subseteq \mathfrak{P}$ be a finite set of
  participants.
  %
  A \emph{(communicating) system over $\ptpset$} is a map
  $\aCS = (\alang_{\p})_{\p \in \ptpset}$ assigning an $\p$-local
  language $\alang_{\p}\neq \Set{\epsilon}$ such that
  $\ptpof[\alang_{\p}] \subseteq \ptpset$ to each participant
  $\p \in \ptpset$.
\end{definition}



By projecting a g-language $\clang$ on a participant $\p$ we obtain
the $\p$-local language describing the sequence of actions performed
by $\p$ in the interactions involving $\p$ in the words of $\clang$.
\begin{definition}[Projection]\label{def:projection}
  The \emph{projection on \p} of an interaction $\gint[][B][@][C]$
  is computed by the function
  $\proj{\_}{\_} : \alfint \times \mathfrak{P} \to \alfact \cup \Set{\epsilon}$ defined by:
  \[
	 \proj{(\gint[])}{A} = \aout \qquad\qquad \proj{(\gint[])}{B} =
	 \ain \qquad\qquad \proj{(\gint[])}{c} = \epsilon
  \]
  %
  and extended homomorphically to interaction words and g-languages.
  %
  The \emph{projection of a g-language $\clang$}, written
  $\proj{\clang}{}$, is the communicating system
  $(\proj{\clang}{\p})_{\p \in \ptpof[\clang]}$.
\end{definition}
\cref{def:projection} recasts in our setting the notion of projection
used, e.g., in~\cite{CarboneHY12,honda16jacm}.



\begin{example}\label{ex:simple}\hfill\\
  Let
  $\clang = \Set{\emptyword,\, \gint[][A][m][D],\,\gint[][A][m][D]\cat\gint[][C][m][B],\, 
                                               \gint[][A][m][B],\, 
                                               \gint[][C][m][B],\,
                                               \gint[][C][m][B]\cat\gint[][A][m][D]}$

  $\proj{\clang}{} = (\proj{\clang}{X})_{\ptp[X] \in
	 \Set{\p,\q,\ptp[c],\ptp[d]}}$ where\\
  $\proj{\clang}{A} = \Set{\varepsilon,\, \aout[A][D],\, \aout[A][B]}$,\\
  $\proj{\clang}{B} = \Set{\varepsilon,\, \ain[C][B]}$, \\
  $\proj{\clang}{C} = \Set{\varepsilon,\, \aout[C][B]}$, and\\
  $\proj{\clang}{D} = \Set{\varepsilon,\, \ain[A][D]}$.
  \finex
\end{example}

\begin{definition}[Split and Join]\label{def:decomp}
\begin{enumerate}[i)]
\item
  The \emph{splitting} operation
  $\ssplit[\_]: \alfint^\infty \to \alfact^\infty$ 
  is the homomorphic extension on words and sets of words of the following function on interactions. 
  \[
	 \ssplit[\gint] = \aout\cat\ain
  \]
  \item
  The \emph{joining} operation on strict synchronous words of actions 
  is the homomorphic extension on words and sets of words of the following function on interactions. 
  \[
	 \join[\aout\cat\ain] = \gint
  \]
  \end{enumerate}
\end{definition}

We define an operation filtering out in an word of action those actions  having a given subject

\begin{definition}[Filtering]\label{def:decomp}
  The \emph{filtering} operation
  $\filter{\_}{\_}: \alfact^\infty \times \mathfrak{P} \to
  \alfint^\infty$ is the homomorphic extension on words of the following function on actions. 
  \[
	 \filter{\aact}{\p} =
	 \begin{cases}
		\aact &  \text{if } \p \text{ is the subject of } \aact
		\\
		\emptyword &  \text{otherwise}
	 \end{cases}
  \]
\end{definition}

\begin{fact}
\label{fac:joinsplit}
Let $\clang$ be a c-language, $\acword\in\clang$, $\aCS$ a communicating system on $\ptpset$, $\px\in\ptpset$, and let $z\in\alfact^\infty$ such that $\ssynch[z]$
\begin{enumerate}[i)]
%\item
%$\join[{\ssplit[{\acword}]}] = \acword$
%\item
%$\ssplit[{\join[{z}]}] = z$
\item
$\filter{{\ssplit[{\acword}]}}{\px} = \proj{\acword}{X}$
\item
\label{fac:joinsplit-feqp}
$\proj{\join[z]}{X} = \filter{z}{\px}$. 

\end{enumerate}
\end{fact}



\subsection{Symmetric synchronous semantics}

We consider a \emph{synchronous} semantics of
communicating systems, similarly to other choreographic approaches
such as~\cite{BravettiZ07,CarboneHY12,Dezani-Ciancaglini16,SeveriD19}.

\begin{definition}[Symmetric synchronous Semantics]\label{def:syncSem}
  Given a system $\aCS$ on $\ptpset$, the set
  \[
	 \ssem{\aCS} = \Set{\acword \in \alfint^\infty \sst  \ptpof[\acword] \subseteq \ptpset \ \land\ \forall \p \in
		\ptpset \qst \ \proj{\acword}{\p} \in \aCS(\p)}
  \]
  is the \emph{{\bf symmetric} (synchronous) semantics} of $\aCS$;
\end{definition}

\subsection{Asymmetric synchronous semantics}

In the asymmetric case we can look at the syncronous communication of the message $\msg$
from $\p$ to $\q$ as composed by the choice of $\aout$ among the possible outputs and the
handshaking communication resulting in the reception of the message by $\q$ ($\ain$).
In the system several different actions performed by participants other that $\p$ and $\q$
can occur in between $\aout$ and $\ain$. The most general asymmetric semantics take into account
such a possibility. 
 

\begin{definition}[Deferred synchronous words]
Let $z\in\alfact^{\infty}$. We say $z$ to be a {\em deferred synchronous word} ($\dsynch[z]$) whenever both the following conditions hold
\begin{enumerate}[a)]
\item 
for each action $\aout$ in $z$, the first action after it of the form $\p\,\_!\_$ or  $\_\,\p?\_$ (if any) is $\ain$;
\item
for each action $\ain$ in $z$, the first action preceding it of the form $\p\,\_!\_$ or  $\_\,\p?\_$ is $\aout$, and it is always the case.
\end{enumerate}
We define $DS=\Set{z\in\alfact^\infty \sst \dsynch[z]}$.
\end{definition}


for each word w in the general semantics there is another word w' in the general semantics
such that w' is synchronous.
So we consider now a semantics modulo equivalence of words like w and w'.



[dare un nome differente a $\comm$ rispetto a chorlang?]

\begin{definition}[Unrestricted asymmetric synchronous semantics]
 \[
	 \gasem{\aCS} = \Set{z \in \alfact^\infty \sst \dsynch[z] \land\ptpof[z] \subseteq \ptpset \ \land\ \forall \p \in
		\ptpset \qst \filter{z}{\p} \in \aCS(\p)  }
  \]
  is the \emph{{\bf unrestricted asymmetric} (synchronous) semantics} of $\aCS$;
\end{definition}

The above unrestricted semantics could be not comfortably handled.
We then restrict it to a sound subset that safely \quo{represents} it. 
The following observation is then at the basis of what will shall use as the 
actual asymmetric synchronous semantics (\cref{def:syncSem} below)

Let us consider a generic trace $z$ of a system with asymmetric synchronous communication 
containing a number of output actions which have not been \quo{matched} yet by their corresponding
input actions. It is not difficult to see that for any $z'$ such that
$z\cat z'$ belongs to the sematics of the system (i.e. $z'$ is a \quo{continuation} of $z$), 
we have that $\hat z \cat z'$ also belongs to the sematics, for any $\hat z$ where the unmatched
output actions appear in a consistent different order.
In particular, in $\hat z$, the unmatched output action can be taken into account only in the final 
part of $z$ 


\begin{definition}[(prefix/strict) synchronous words]
% We say that $\aact[a]\cat\aact[b]\in\alfact^2$ is a split-pair of action
%  whenever $\aact[a]=\aout$ implies $\ain$.
%  
  A word of actions $z\in\alfact^{\infty}$ is 
  \begin{enumerate}[i)]
  \item
  {\em strictly synchronous} $(\ssynch[z])$ whenever
   $z=\ssplit[w]$ for some $\acword\in\alfint^\infty$.
  \item
  {\em prefix synchronous} $(\psynch[z])$ whenever
  \begin{description} 
  \item
  If $z\in\alfact^{*}$ then $z=z'\cat o$ where $\ssynch[z']$ and $o$ is a word of output actions;
  \item
  If $z\in\alfact^{\omega}$ then there exists a finite word $o$ of output actions  such that 
  $\ssynch[{z\setminus o}]$
  \end{description}
\end{enumerate}
  $\slang\subseteq \alfact^{\infty}$  is a \emph{strictly ({\text{resp.} prefix) synchronous} language
  of actions} (\emph{ss-(ps-)language} for short) whenever each word in $\slang$ is strictly (resp. prefix) synchronous.
  \end{definition}
  
  We define a relation on deferred synchronous words.
\begin{definition}[$\commds$]
The relation $\commds\ \subseteq DS{\times}DS$ is defined by
$$z\commds z'  \quad \text{if} \quad \forall \px.\ \filter{z}{\px}=\filter{z}{\px}$$
\end{definition}

  
\begin{lemma}
Let be $\aCS$ a communicating system.
\begin{enumerate}[i)]
\item
Given $z\in \gasem{\aCS}$, there exists $z'\commds z$ such that $\psynch[z']$ and $z'\in \gasem{\aCS}$.
\item
$\gasem{\aCS}$ is $\comm$-closed.
\end{enumerate}
\end{lemma}
 

\begin{definition}[Asymmetric synchronous Semantics]\label{def:syncSem}
  Given a system $\aCS$ on $\ptpset$, the set
  \[
	 \asem{\aCS} = \Set{z \in \alfact^\infty \sst \ptpof[z] \subseteq \ptpset \ \land\ \psynch[z]\  \land\ \forall \p \in
		\ptpset \qst \filter{z}{\p} \in \aCS(\p)  }
  \]
  is the \emph{{\bf asymmetric} (synchronous) semantics} of $\aCS$;
\end{definition}

\begin{lemma}
Let $\aCS$ be a system on $\ptpset$.
$$\gasem{\aCS} = \Set{z\in\alfact^\infty \sst  \psynch[z] \land \exists z'\in\asem{\aCS}.\  z\commds z'}$$ 
\end{lemma}

[say that the two semantics are a g- and an s-language, that is they are prefix-closed]

\begin{lemma}
\label{fac:asymmsubsymm}
Let  $\aCS$ be a communicating system on $\ptpset$.
\begin{enumerate}[i)]
\item
$\ssem{\aCS} =  
\Set{\acword \in \alfint^\infty \sst  
\ptpof[\acword] \subseteq \ptpset \ \land\ \forall \p \in \ptpset \qst \filter{\ssplit[\acword]}{\p} \in \aCS(\p)}$.
\item
\label{fac:asymmsubsymm-aimps}
$z\in\asem{\aCS}$ and $\ssynch[z]$ implies $\join[z]\in\ssem{\aCS}$.
\item
\label{fac:asymmsubsymm-asubs}
$\pref[{\ssplit[{\ssem{\aCS}}]}] \subseteq \asem{\aCS}$.
\end{enumerate}
\end{lemma}

\begin{example}
\label{ex:simple2}
The two semantics are non equivalent in general.
Let us consider $\aCS = ({\alang}_{\px})_{\ptp[X] \in \Set{\p,\q}}$ where\\
\centerline{$\alang_{\p}= \Set{\aout[A][B][][m], \aout[A][B][][y]}$ \qquad $\alang_{\p}= \Set{\ain[A][B][][m]}$ }
is clearly lock-free with the symmetric semantics and non lock-free with the asymmetric one.

 Even in case we restrict to systems
obtained by projecting CUI c-languages the two semantics can differ.
Let us consider the g-language $\clang$ in \cref{ex:simple}.
$\clang$ is CUI, hence we have 
  $$\ssem{\proj{\clang}{}} = \clang$$ 
whereas 
  $$\asem{\proj{\clang}{}} = \ssplit[\clang] \cup \pref[\Set{\aout[A][B][][m]\cat\ain[A][B][][m]\cat\aout[C][B][][m]}]$$
\finex
\end{example} 

\begin{definition}[Asymmetrically-synchronous correctness and completeness]\label{def:cc}
  Let $\aCS$ be a communicating system and $\clang$ a g-language.\\
  $\aCS$ is \emph{a-synch correct (resp. complete) with respect to
    $\clang$} if $\join[{\asem{\aCS}}]\subseteq \clang$ (resp. 
  $\join[{\asem{\aCS}}]\supseteq \clang$).
\end{definition}


\begin{definition}[CUIasym]
\label{def:cuiasym}
A c-language $\clang$ on $\ptpset$ is CUIasym $(\cuuiasym[\clang])$ whenever,
 for each $\p,\q\in\ptpset$ and $\acword, \acword_1\in\clang$, the following holds:\\
 \centerline{$\proj{\acword}{A} = \proj{\acword_1}{A}\ \land\  \proj{\acword}{B} = \proj{\acword_2}{B}\ \land\  \acword_1\cat\gint\in\clang\ \implies\ \acword\cat\gint\in\clang$}
\end{definition}

\begin{proposition}
Let $\clang$ be a c-language.\\
\centerline{$\cuuiasym[{\clang}]\ \implies\ \cuui[\clang]$}
\end{proposition}
\begin{proof}
Let us assume $\cuuiasym[{\clang}]$ to hold. In order to show  $\cuui[{\clang}]$, 
by definition let us consider $\acword_1,\acword_2,\acword\in\clang$, $\p,\q,\msg$ such that
$$\proj{\acword}{A} =\proj{\acword_1}{A}\quad \proj{\acword}{B} =\proj{\acword_1}{B}
\quad \acword_1\cat\gint, \acword_2\cat\gint\in\clang$$
From $\proj{\acword}{A} =\proj{\acword_1}{A}$ and $\acword_1\cat\gint\in\clang$,
by $\cuuiasym[{\clang}]$ we immediately get  $\acword\cat\gint\in\clang$, that is what we needed to
prove $\cuui[\clang]$.
\qed
\end{proof}


\begin{theorem}[Equivalence of semantics]
\label{th:eqsem}
 Let $\clang$ be a CUI c-language.
$$
\cuuiasym[\clang] \quad \text{iff}\quad \asem{\proj{\clang}{}} =  \pref[{\ssplit[\ssem{\proj{\clang}{}}]}]\quad (=
 \pref[{\ssplit[\clang]}] \text{ by } \cuui)
$$
\end{theorem}
\begin{proof}
($\Leftarrow$).
By contraposition, let us assume $\cuuiasym[\clang]$ not to hold. In that case, by \cref{def:cuiasym},
since we have $\cuui[\clang]$, it has to be the case that
there exists $\p,\q\in\ptpset$ and $\acword, \acword_1\in\clang$ such that
 $$
 \proj{\acword}{A} = \proj{\acword_1}{A}\ \land\  \acword_1\cat\gint\in\clang\ \land\ \acword\cat\gint\not\in\clang
 $$
 This necessarily implies $\acword_1\neq\acword$.
We have now that
 $\proj{(\acword_1\cat\gint)}{A}\in{\proj{\clang}{A}}$.
 So, using  \cref{fac:asymmsubsymm} and the above,
 $\proj{(\acword_1\cat\gint)}{A}
= \proj{\acword_1}{A}\cat\aout[A][B][][m]
= \proj{\acword}{A}\cat\aout[A][B][][m]
 = \filter{(\ssplit[{\acword}])}{\p}\cat\aout[A][B][][m]
 = \filter{(\ssplit[{\acword}]\cat\aout[A][B][][m])}{\p}
 \in{\proj{\clang}{A}}$.
 Now, since, for each $\px\neq\p$ we have $\proj{({\acword})}{\px}=\filter{(\ssplit[{\acword}])}{\px}$,
 by definition of $\asem{\proj{\clang}{}}$ we get 
 $$\ssplit[{\acword}]\cat\aout[A][B][][m]\in\asem{\proj{\clang}{}}.$$
 In order to get the thesis it is then enough to show that 
 $$\ssplit[{\acword}]\cat\aout[A][B][][m] \not\in\pref[{\ssplit[\ssem{\proj{\clang}{}}]}].$$ 
 This means that we have to show that for no $u\in\alfact^\infty$ such that
 $\ssynch[{\ssplit[{\acword}]\cat\aout[A][B][][m]\cat u}]$
 we have $\join[{\ssplit[{\acword}]\cat\aout[A][B][][m]\cat u}]\in\ssem{\proj{\clang}{}}$. 
 If there were such an $u$, it should have necessarily the form $\ain[A][B][][m]\cat u'$.
 This, by prefix closure of $\ssem{\proj{\clang}{}}$ (=$\clang$), would imply
$\join[{\ssplit[{\acword}]\cat\aout[A][B][][m]\cat \ain[A][B][][m]}]\in\ssem{\proj{\clang}{}}$,
that is $\acword\cat\gint\in\ssem{\proj{\clang}{}}$ and hence, by $\cuui$,
$\acword\cat\gint\in{{\clang}}$, contradicting our hypothesis.\\
%
 ($\Rightarrow$).
 We assume $\cuuiasym$. Now, to show the thesis by contradiction, we assume 
 \begin{equation}
 \label{eq:aneqs}
 \asem{\proj{\clang}{}} \neq  \pref[{\ssplit[\ssem{\proj{\clang}{}}]}].
 \end{equation}
 Since, by \ref{fac:asymmsubsymm-asubs}(\ref{fac:asymmsubsymm}),
 $\pref[{\ssplit[{\ssem{\proj{\clang}{}}}]}] \subseteq \asem{\proj{\clang}{}}$
 we have that  (\ref{eq:aneqs}) holds in case there exists $\hat z \in\alfact^*$ such that
 \begin{equation}
 \label{eq:negass}
 \ssynch[\hat z]\ \land\ \exists \p,\q,\msg[m].\  \hat z\cat\aout \in \asem{\proj{\clang}{}} \land\  \join[{\hat z\cat\aout\cat\ain}]\not\in\ssem{\proj{\clang}{}}
 \end{equation}
 By definition of $\asem{\proj{\clang}{}}$, 
for each $\px\in\ptpof[\hat z\cat\aout]$  there is a $z_{\px}\in\proj{\clang}{X}\cap\alfact^*$ such that
$\filter{(\hat z\cat\aout)}{\px}=z_{\px}$. In particular, $z_{\p}= \filter{\hat z}{\p}\cat\aout$ and,
for  $\px\neq\p$, $\filter{\hat z}{\px}=z_{\px}$.
Now, by definition of projection,  for each $\px\in\ptpof[z]$ we can set $\acword_{\px}\in\clang$
as a word such that  $\proj{\acword_{\px}}{X}= z_{\px}$. 
Since $z_{\p}= \filter{\hat z}{\p}\cat\aout$, we can infer that  
$\acword_{\p}= \acword_1\cat\gint$ for some $\acword_1\in\clang$ such that
$\proj{\acword_1}{A}= \filter{\hat z}{\p}$. 
 
 We now observe that  $\asem{\proj{\clang}{}}$ is prefix closed, 
so $\hat z\in \asem{\proj{\clang}{}}$. 
Hence, by $ \ssynch[\hat z]$ and  
\cref{fac:asymmsubsymm}(\ref{fac:asymmsubsymm-aimps}), 
$\join[\hat z]\in\ssem{\proj{\clang}{}}$. Then,
since $\clang$ is CUI, we get $\join[\hat z]\in\clang$. 
Moreover, by Fact \ref{fac:joinsplit}(\ref{fac:joinsplit-feqp}) for each $\px$,
$\proj{\join[\hat z]}{X}=\filter{\hat z}{\px}$. 
We now proceed by noticing that $\acword_1\neq\join[\hat z]$. In fact, otherwise, we would have $\join[\hat z]\cat\gint\in\clang$
and hence, by $\cuui$, 
${\join[\hat z]}\cat\gint\in\ssem{\proj{\clang}{}}$, 
that is $\join[{\hat z\cat\aout\cat\ain}]\in\ssem{\proj{\clang}{}}$,
contradicting the third conjunct of our assumption (\ref{eq:negass}).\\
 Now, by setting $\acword=\join[\hat z]$, by the above
we have that
\begin{itemize}
\item
$\proj{\acword_1}{A} = \proj{\acword}{A}$\qquad 
(since $\proj{\acword_1}{A}=\filter{\hat z}{\p}=\proj{\join[\hat z]}{A}=\acword$)
%\item 
%$\proj{\acword_2}{B} = \proj{\acword}{B}$
\item
$\acword_1\cat\gint\in\clang$
\end{itemize} 
So, by $\cuuiasym[\clang]$, $\acword\cat\gint=\join[\hat z]\cat\gint\in\clang$. 
This, as previously shown, would imply  $\join[{\hat z\cat\aout\cat\ain}]\in\ssem{\proj{\clang}{}}$ that contradicts the third conjunct of (\ref{eq:negass}).
\qed
 \end{proof}
 
 \begin{example}
In \cref{ex:simple2} we have seen that the two synchronous semantics differ on the g-language 
$\clang$ of \cref{ex:simple}.
In fact it is possible to show that $\cuuiasym[{\clang}]$ does to hold.
By taking $\acword_1=\gint[][A][m][D]$ and $\acword=\gint[][A][m][B]$, we have that
$$\proj{\acword_1}{C}=\emptyword = \proj{\acword}{C} \quad\text{ and }\quad \acword_1\cat\gint[][C][m][B]\in\clang$$
However $\gint[][A][m][B]\cat\cat\gint[][C][m][B]= \acword\cat\cat\gint[][C][m][B]\not\in\clang$.
\finex
\end{example}

 
 It is easy to show that in the  characterization of \cref{th:eqsem}  one cannot get rid of the CUI hypothesis.
 In fact, for $\clang=\Set{\gint[][A][m][B], \gint[][C][m][D]}$, the two semantics are
 trivially equivalent on $\proj{\clang}{}$ and $\cuui[\clang]$ does not hold.
 In any case, a chracterisation of the equivalence for c-labguages that could be non CUI,
 would be hardly on any use.
 

 
% ========================
% 
%By definition of semantics (\cref{def:syncSem}), it is enough to show that 
%the set
%$$
%Y = \Set{\hat u   \sst \ssynch[\hat u]\ \land\ \exists \p,\q,\msg[m].\  \hat u\cat\aout \in \asem{\proj{\clang}{}} \land\  \join[{\hat u\cat\aout\cat\ain}]\in\ssem{\proj{\clang}{}} }
%$$
%is empty.
%By contradiction let us assume $Y$ not to be empty.
%We can then take a word $\hat z\in Y$, namely such that $\ssynch[\hat z]$,
%$\hat z\cat\aout\in \asem{\proj{\clang}{}}$ but $\hat z\cat\aout\cat\ain\not\in\asem{\proj{\clang}{}}$.
%By definition of $\asem{\proj{\clang}{}}$, 
%for each $\px\in\ptpof[\hat z\cat\aout]\subseteq\ptpset$  there is a $z_{\px}\in\proj{\clang}{X}\cap\alfact^*$ such that
%$\filter{(\hat z\cat\aout)}{\px}=z_{\px}$. In particular, $z_{\p}= \hat z'_{\p}\cat\aout$ 
%for some $\hat z'_{\p}$ and, if $\px\neq\p$, $\filter{\hat z}{\px}=z_{\px}$.
%Now, by definition of projection,  for each $\px\in\ptpof[z]$ we can set $\acword_{\px}\in\clang$
%as a word such that  $\proj{\acword_{\px}}{X}= z_{\px}$. 
%Since $z_{\p}= \hat z'_{\p}\cat\aout$, we can infer that  
%$\acword_{\p}= \acword_1\cat\gint$ for some $\acword_1\in\clang$ such that
%$\proj{\acword_1}{A}= \hat z'_{\p}$. 
%
%We now observe that  $\asem{\proj{\clang}{}}$ is prefix closed, 
%so $\hat z\in \asem{\proj{\clang}{}}$. 
%Hence, by Fact \ref{fac:asymmsubsymm}, $\join[\hat z]\in\ssem{\proj{\clang}{}}$ and then,
%since $\clang$ is CUI, we get $\join[\hat z]\in\clang$. 
%Moreover, for $\px\neq\p$,
%$\proj{\join[\hat z]}{X}=\filter{\hat z]}{\px}$. 
%%So, in particular $\proj{\join[\hat z]}{B}=\filter{\hat z}{\q}=z_{\q}=\proj{\acword_{\q}}{B}$.
%We notice that $\acword_1\neq\join[\hat z]$. In fact, otherwise, we would have $\join[\hat z]\cat\gint\in\clang$
%and hence, by $\cuui$ and Fact \ref{fac:asymmsubsymm}, $\ssplit[{{\join[\hat z]}\cat\gint}]\in\asem{\proj{\clang}{}}$, that is $\hat z\cat\aout\cat\ain\in\asem{\proj{\clang}{}}$,
%contradicting what we had assumed. 
%
%
%
%
%Now, by setting $\acword=\join[\hat z]$ % and $\acword_2=\acword_{\q}$
%we have that
%\begin{itemize}
%\item
%$\proj{\acword_1}{A} = \proj{\acword}{A}$
%%\item 
%%$\proj{\acword_2}{B} = \proj{\acword}{B}$
%\item
%$\acword_1\cat\gint\in\clang$
%\end{itemize} 
%So, by CUI2, $\acword\cat\gint\in\clang$. 
%This, as shown above, would imply by definition implies $\hat z\cat\aout\cat\ain\in\asem{\proj{\clang}{}}$, so contradicting that $\hat z\in Y$.
%\qed
%
%Let us now consider $\hat z=\aact[a]_0\cat\ldots\cat\aact[a]_{n-1}$. By definition it follows that
%$\hat z \in \ssem{\proj{\clang}{}}$. 
%Moreover, for each $\px\in\ptpof[z]\subseteq\ptpset$  there is a
%subsequence $z_{\px}$ of $\hat z$ of the form $\aact[a]_{j_{\px}0}\cat\ldots\cat\aact[a]_{j_{\px}m_{\px}}$ belonging to $\proj{\clang}{X}$ such that $\hat z \in \shuffle_{\ptpof[z]}z_{\px}$.
%
%For each $\px\in\ptpof[z]$ let now $\acword_{\px}\in\clang$ be a word such that $\proj{\acword}{X}= z_{\px}$. 
%By definition of projection it follows that $\acword_{\p}$ has the form $\acword'\cat\gint$ 
%for some $\acword'$.
%By Lemma XXX 
% This immediately implies that $\acword_{\q}$ has the form $\acword''\cat\gint$ for some $\acword''$.
% 
% 
%
%Let now be $\acword = \acword_1\cat\gint$ the word in \\
%$(\subseteq)$



\begin{corollary}
Let $\clang$ be a CUIasym g-language. Then $\proj{\clang}{}$ is a-synch correct and complete for $\clang$.
\end{corollary}


It is now possible to show that if we consider the calculus of MultiParty Sessions of [DezaniSeveri]
(restricted to sequential reductions), on any system obtained by projecting a well-formed 
global types the two synchronous semantics are equivalent.


\begin{proposition}
For each global type ${\mathsf G}$,\\
\centerline{${\mathsf G}$ well-formed\ $\implies$\ $\cuuiasym[{\clang({\mathsf G})}]$.}
\end{proposition}

\begin{proof}
Let ${\mathsf G}$ be well-formed.
By contradiction, let us assume $\cuuiasym[{\clang({\mathsf G})}]$ not to hold.
This means that there exist $\acword_1,\acword\in\clang({\mathsf G})$ such that, for a certain
interaction $\gint$,
$$
\proj{\acword_1}{A} = \proj{\acword}{A}      \qquad      \acword_1\cat\gint\in\clang({\mathsf G})   
\qquad \acword\cat\gint\not\in\clang({\mathsf G})
$$
By well-formedness, each continuation of $\acword$ must contain an interaction with participant $\p$.
Let us then consider $\acword$ with one of its  continuation that is a word of the form
$$
\acword\cat\acword'\cat \aint \qquad \text{where } \p\in\ptpof[\aint] \quad \text{and }\proj{\acword'}{A}=\emptyword
$$
Now, since each participant has to distinguish each pair of maximal branches by two inputs,
we have necessarily that $\aint=\gint$. Moreover, since  $\proj{\acword'}{A}=\emptyword$,
also $\acword\cat \aint\cat\acword'\in\clang({\mathsf G}) $, contradicting our assumption
$\acword\cat\gint\not\in\clang({\mathsf G})$
\qed
\end{proof}


%\begin{proposition}[FALSE]
%Let $\clang$ be a  g-language. 
%$$\cuuiasym[\clang] \implies ba[\clang]$$
%\end{proposition}
%\begin{proof}
%Let us assume $\cuuiasym[\clang]$. By contradiction, let us assume also $ba[\clang]$ not to hold.
%So, by definition, we have two maximal words $\acword_1, \acword_2\in\clang$ and a participant
%$\p$ such that $\proj{\acword_1}{A}\prec \proj{\acword_2}{A}$.
%Let now $\hat\acword_1$ and $\hat\acword_2$ be the longest prefixes of, respectively
%$\acword_1$ and  $\acword_2$ such that $\proj{\hat\acword_1}{A}= \proj{\hat\acword_2}{A}$.
%Since $\proj{\acword_1}{A}\prec \proj{\acword_2}{A}$ there exists a prefix of $\acword_2$ 
%of the form $\hat\acword_2\hat\acword'_2\cat\gint$, 
%where $\proj{\hat\acword'_2}{A}=\emptyword$. 
%Since $\proj{\hat\acword_2\hat\acword'_2}{A}=\proj{\hat\acword_1}{A}$
%\end{proof}


[Deadlock Freeness in the asymmetric case cannot be formalised as in the symmetric one.]


Two interactions $\aint$ and $\aint[b]$ are \emph{independent} (in
symbols $\intpar$)
when % their sets of participants are disjoint, namely
$\ptpof[\aint] \cap \ptpof[{\aint[b]}] = \emptyset$.
%
Informally, a g-language is \emph{concurrency closed} when
\quo{independent} interactions can be swapped.
%
The concurrency closure on infinite words is delicate.
%
One in fact has to allow infinitely many swaps while avoiding that
they make an interaction disappear by \quo{pushing it infinitely
  down}.
%
Technically, we consider Mazurkiewicz's traces~\cite{maz86} on
$\alfint$ with independence relation $\intpar$:
\begin{definition}[Concurrency closure]\label{def:cc}
  Let $\comm$ be the reflexive and transitive closure of the relation
  $\equiv$ on finite interaction words defined by
  $\acword\,\aint\,\aint[b]\,\acword' \equiv
  \acword\,\aint[b]\,\aint\,\acword'$ where $\intpar$.
  %
  Following \emph{\cite[Def. 2.1]{Gastin90}}, $\comm$ extends to
  $\alfint^\omega$ by letting $\acword,\acword'\in\alfint^\omega$ and defining
  \[
	 \acword \comm\acword' \quad\iff\quad \acword \ll\acword' \qand \acword' \ll\acword
  \]
  %
  where
  $\acword \ll \acword'$ iff for each finite prefix $\acword_1$ of
  $\acword$ there are a finite prefix $\acword_1'$ of $\acword'$ and a
  g-word $\hat \acword \in \alfint^\star$ such that
  $\acword_1 \cat \hat \acword \comm \acword_1'$.
  %
  A g-language $\clang$ is \emph{concurrency closed} (c-closed for
  short) if it coincides with its concurrency closure, namely
  $\clang = \Set{\acword \in \alfint^\infty \sst \exists
	 \acword'\in\clang \qst \acword \comm \acword'}$.
\end{definition}

Semantics of systems are naturally c-closed since in a distributed setting
independent events can occur in any order.
%
Indeed
\begin{toappendix}
The relation $\comm$ can be characterised as follows.
  \begin{lemma}\label{lemma:equalProj}
  Given a g-language $\clang$ and two words
  $\acword_1, \acword_2 \in \clang$, $\acword_1 \comm \acword_2$ iff
  $\proj{\acword_1}{\p} = \proj{\acword_2}{\p}$ for each
  $\p \in \ptpof[\clang]$.
\end{lemma}
%\equalProj*
\end{toappendix}
\begin{appendixproof}
  This follows directly from~\cite[Proposition 2.2]{Gastin90}.
  \qed
\end{appendixproof}

\begin{restatable}{proposition}{ccsem}\label{prop:par}
  Let $\aCS$ be a system. Then $\ssem{\aCS}$ is
  c-closed.
\end{restatable}
\begin{toappendix}
  \ccsem*
\end{toappendix}
\begin{appendixproof}
  Trivial, since closure under swap does not change the projection by
  \cref{lemma:equalProj}.
  \qed
\end{appendixproof}

% \paragraph*{The ($\proj{(\_)}{}$,$\ssem{\_}$) Galois connection\footnote{We thank an anonymous referee for the introduction of this paragraph.}}



%%% Local Variables:
%%% mode: latex
%%% TeX-master: "main"
%%% TeX-master: "main"
%%% TeX-master: "main"
%%% End:
