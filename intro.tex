
For what concerns synchronous communications, it is actually possible to consider different synchronous models.
There is an extensive literature on process algebras (eg CCS, pi-calculus, ACP, etc.) where the symmetry between sender and receiver in synchronous communications is assumed when resolving choices. In such models, no essential distinction is made between server and receiver,  so leading to situations where a receiver can force a specific choice on a sender, as happens also in  \cite{BLT20b}.
In a blocking message exchange however, the sender and receiver can have different roles during the communication and this reflects to other aspects of the model, in particular internal and external choices. This distinction is obviously fundamental for asynchronous communication but, in many cases, this distinction is also kept for synchronous communication, as done, for instance in \cite{aey03}.
 In this synchronous model,  communication means the sender chooses to send a message and gets blocked until the message is received. However, the send/choice is already fixed by the time the receive happens\footnote{We thank an anonimous referee of another paper for implicitly suggesting
 us to complete the investigation on the \quo{preservation of communication-properties by composition}
by taking into account also  asymmetric synchronous interactions.}.

We can consider the following simple system for (finite) communicating processes in the style of MPST
as a formalism for describing 
examples for the two above possible synchronous communications. 

 Let $P$ range over the terms of the following grammar:
  \begin{align*}
	 \begin{array}{lll@{\qquad\qquad}lll}
		P   &   ::=  & \mathbf{0}\   \mid\  \q!\Set{\al_1:P_1\ldots \al_n:P_n }
		\ \mid\  \q?\Set{\al_1:P_1\ldots \al_n:P_n }
	 \end{array}
  \end{align*}
  where the  $\ll_i$ are pairwise distinct messages. A term $P$ denotes a process.
  %
 
   A term $\mathcal{M}$ generated by the grammar
	 \begin{align*}
	 \mathcal{M}  ::={\p}\triangleright  P\ \mid\  \mathcal{M} \mid \mathcal{M}
	 \end{align*}
  is a \emph{system of named processes}
  
As discussed previously, two synchronous operational semantics are possible, that we shall dub
{\em symmetric} and {\em asymmetric}. For our simple calculus they are naturally as follows.


\noindent
{\sf Symmetric-sinchronous operational semantics}
$$
\begin{array}{r}
\mathcal{M} \mid {\p}\triangleright  \text{\bf !}\Set{\q:\msg.P}{\cup}\Set{\q_i:\msg_i.P'_i}_{i\in I}
\mid {\q}\triangleright \text{\bf ?}\Set{\p:\msg.Q}{\cup}\Set{\msg_j:\al_j.Q'_j}_{j\in J}  
\\[2mm]
\arro{\gint[][A][m][B]} \
\mathcal{M} \mid {\p}\triangleright  P \mid  {\q}\triangleright Q 
\end{array}
$$
\vspace{4mm}
\noindent
{\sf Asymmetric-sinchronous operational semantics}
$$
\begin{array}{rcl}
\mathcal{M} \mid {\p}\triangleright  \text{\bf !}\Set{\q:\al.P}{\cup}\Set{\q_i:\al_i.P'_i}_{i\in I}
& \arro{\aout} &
\mathcal{M} \mid {\p}\triangleright  \q\text{\bf !}[\msg].P
\\[2mm]
\mathcal{M} \mid {\p}\triangleright   \q\text{\bf !}[\msg.P] \mid 
{\q}\triangleright ?\Set{\p:\al.Q}{\cup}\Set{\p_j:\al_j.Q'_j}_{j\in J}  
& \arro{\ain} &
\mathcal{M} \mid {\p}\triangleright  P \mid  {\q}\triangleright Q 
\end{array}
$$

It is not difficult to check that the two semantics are non equivalent, since it is enough to have a 
simple system where a participant has an input (or output) capability which is not matched
in the sender (or receiver).
$$
{\p}\triangleright  \text{\bf !}\Set{\ptp[B]{:}\msg.\mathbf{0},\, \ptp[B]{:}\msg[n].\mathbf{0}}
\mid 
{\q}\triangleright  \text{\bf ?}\Set{\ptp[A]{:}\msg.\mathbf{0}}
$$
By the symmetric semantics the only reduction sequence is
$$
{\p}\triangleright  \text{\bf !}\Set{\ptp[B]{:}\msg.\mathbf{0},\, \ptp[B]{:}\msg[n].\mathbf{0}}
\mid 
{\q}\triangleright  \text{\bf ?}\Set{\ptp[A]{:}\msg.\mathbf{0}}
\arro{\gint[][A][m][B]}
\mathbf{0} \mid \mathbf{0}
$$
whereas, in the asymmetric one the system can get stuck even if some process has not completed (i.e. reduced to $\mathbf{0}$)

$$
{\p}\triangleright  \text{\bf !}\Set{\ptp[B]{:}\msg.\mathbf{0},\, \ptp[B]{:}\msg[n].\mathbf{0}}
\mid 
{\q}\triangleright  \text{\bf ?}\Set{\ptp[A]{:}\msg.\mathbf{0}}
\arro{\aout[A][B][][n]}
{\p}\triangleright  \ptp[B]\text{\bf !}[\msg[n]].\mathbf{0}
\mid 
{\q}\triangleright  \text{\bf ?}\Set{\ptp[A]{:}\msg.\mathbf{0}}
\not\!\longrightarrow
$$

In order to compare the two semantics some natural relation has to be formalised,
making equivalent sequences like $\gint[][A][m][B]\cat\gint[][C][n][D]$
and $\aout\cat\ain\cat\aout[C][D][][n]\cat\ain[C][D][][n]$.
As well as $\aout\cat\ain\cat\aout[C][D][][n]\cat\ain[C][D][][n]$ and $\aout\cat\aout[C][D][][n]\cat\ain\cat\ain[C][D][][n]$.


The different behaviours in the above example depends immediately on the fact that a participant
possesses an output action and no participants contains the corresponding input.

This problem never arise if, as in many choreographic approaches to the development of concurrent systems (MPTSs, for instance), the latter
are obtained by means of a projection operation on choreogreaphies, as done in MPSTs.
A global type formalism for our example can be

$$\mathsf{G} ::= \mathbf{end} \mid  \bigvee_{i\in I}\Set{\p_i\rightarrow\q_i{:}\msg[m]_i.\mathsf{G}_i}$$

In any MPST setting (any choreographic setting as a matter of facts) the least property
expected is that a system obtained by projection do behave as described by the global type.
It is then reasonable to restrict our investigation to global descriptions whose
projected system - with the symmetric semantics -  behaves as the global description prescribes.
For instance, Let us consider the following global type.
$$
\mathbf{G} = \bigvee\Set{\p\rightarrow\ptp[D]{:}\msg[m].\ptp[C]\rightarrow\ptp[B]{:}\msg[m].\mathbf{end},\,
                                         \p\rightarrow\ptp[B]{:}\msg[m].\mathbf{end},\,
                                         \ptp[C]\rightarrow\ptp[B]{:}\msg[m].\p\rightarrow\ptp[D]{:}\msg[m].\mathbf{end}}
$$

The system obtained by projection is


$$
\proj{\mathbf{G}}{} = {\p}\triangleright  \text{\bf !}\Set{\ptp[D]{:}\msg.\mathbf{0},\, \ptp[B]{:}\msg.\mathbf{0}}
\mid 
{\q}\triangleright  \text{\bf ?}\Set{\ptp[C]{:}\msg.\mathbf{0},\, \ptp[A]{:}\msg.\mathbf{0}}
\mid
{\ptp[C]}\triangleright  \text{\bf !}\Set{\ptp[B]{:}\msg.\mathbf{0}}
\mid
{\ptp[D]}\triangleright  \text{\bf ?}\Set{\ptp[A]{:}\msg.\mathbf{0}}
$$

with the symmetric semantics, 
the only possible reduction sequences are
$$
\proj{\mathbf{G}}{} \arro{\gint[][A][m][D]}
{\q}\triangleright  \text{\bf ?}\Set{\ptp[C]{:}\msg.\mathbf{0},\, \ptp[A]{:}\msg.\mathbf{0}}
\mid
{\ptp[C]}\triangleright  \text{\bf !}\Set{\ptp[B]{:}\msg.\mathbf{0}}
\arro{\gint[][C][m][B]}
\mathbf{0}
$$
or
$$
\proj{\mathbf{G}}{} 
\arro{\gint[][C][m][B]}
{\p}\triangleright  \text{\bf !}\Set{\ptp[D]{:}\msg.\mathbf{0},\, \ptp[B]{:}\msg.\mathbf{0}}
\mid
{\ptp[D]}\triangleright  \text{\bf ?}\Set{\ptp[A]{:}\msg.\mathbf{0}}
\arro{\gint[][A][m][D]}
\mathbf{0}
$$
or
$$
\proj{\mathbf{G}}{} \arro{\gint[][A][m][B]}
{\ptp[C]}\triangleright  \text{\bf !}\Set{\ptp[B]{:}\msg.\mathbf{0}}
\mid
{\ptp[D]}\triangleright  \text{\bf ?}\Set{\ptp[A]{:}\msg.\mathbf{0}}
\not\longrightarrow
$$

That is, exactly as described by $\mathbf{G}$.
Such a property, is usually obtained in MPTSs formalisms by means of strong restrictions
on the syntax (forcing the same sender in the first interactions in a disjunction)
and requirements on the processes obtained by projection.
In our example the htwo syntax differ on  $\proj{\mathbf{G}}{}$.


As a matter of fact 
{\small
$$
\begin{array}{rcl}
\proj{\mathbf{G}}{} 
           & \arro{\aout[A][B][][m]} &
{\p}\triangleright  \q\text{\bf !}[\msg].\mathbf{0}
\mid 
{\q}\triangleright  \text{\bf ?}\Set{\ptp[C]{:}\msg.\mathbf{0},\, \ptp[A]{:}\msg.\mathbf{0}}
\mid
{\ptp[C]}\triangleright  \text{\bf !}\Set{\ptp[B]{:}\msg.\mathbf{0}}
\mid
{\ptp[D]}\triangleright  \text{\bf ?}\Set{\ptp[A]{:}\msg.\mathbf{0}}\\
      &   \arro{\ain[A][B][][m]}&
{\ptp[C]}\triangleright  \text{\bf !}\Set{\ptp[B]{:}\msg.\mathbf{0}}
\mid
{\ptp[D]}\triangleright  \text{\bf ?}\Set{\ptp[A]{:}\msg.\mathbf{0}}\\
      &   \arro{\aout[C][B][][m]} &
{\ptp[C]}\triangleright  \ptp[B]\text{\bf !}[\msg].\mathbf{0}
\mid
{\ptp[D]}\triangleright  \text{\bf ?}\Set{\ptp[A]{:}\msg.\mathbf{0}}\\
   &\not\!\!\longrightarrow &
\end{array}
$$
}

In our simplified setting, our investigation could be worded as:
Is it possible to characterise (symmetrically) correct and complete global types 
such that on their projections the two semantics are (modulo some equivalence of the sort
mentioned before) equivalent?
This problem is can be also rephrased as: is it possible to characterise
(asymmetrically) correct and complete global types?
A really general solution, however, should not work on the choreographic model of MPTSs only.
We shall then consider the choreographic framework of FCL.






















%%% Local Variables:
%%% mode: latex
%%% TeX-master: "main"
%%% TeX-master: "main"
%%% End:
